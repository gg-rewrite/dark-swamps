\documentclass[10pt,twoside]{report}

\usepackage[utf8]{inputenc}
\usepackage[T1,T2A]{fontenc}
\usepackage[english,russian]{babel}
\usepackage[a6paper,top=1cm,bottom=1.5cm,left=1cm,right=1cm]{geometry}
\usepackage[inline]{enumitem}
\usepackage{graphicx}
\usepackage{float}
\usepackage{pbox}

\setdescription{labelindent=1cm, leftmargin=\parindent}

\begin{document}
% Title, band, bandmembers
\thispagestyle{empty}
\begin{center}
{\LARGE \textbf{Отряд}}

\end{center}
{\LARGE

  Название:\hfill

  \vfill
  
\begin{tabular}{p{1cm} p{2cm} c}
   & Имя & Профессия \\
  1. & & \\
  2. & & \\
  3. & & \\
  4. & & \\
  5. & & \\
  6. & & \\
\end{tabular}
}
\vfill
\pagebreak

% Description, stats
\section*{Информация}
\begin{description}[noitemsep]
\item[Название отряда]:{\slshape Ястребы пограничья, Братья плакучей
    ивы, Общество луны и ключа, Рубайлы из Заславля, Черные стрелы,
    Стальные гончие, Ржавые легионеры, Дикие вепри, Разящие клинки,
    Стражи старой башни, Змееборцы, Хранители дуба, Братство
    св. Дитмара, Тени из Пинеска\ldots}
  \\ [2ex]
\item[Тип отряда и его цели]:
  \begin{itemize}[noitemsep]
    \item Наемники--- охранять или убивать людей за
      деньги.
    \item Охотники на чудовищ--- убивать опасных зверей и болотных
      чудовищ.
    \item Воины Церкви--- карать еретиков и распространять веру.
    \item Расхитители гробниц--- разорять древние руины и курганы.
    \item Мистический ковен--- искать артефакты и проводить ритуалы.
    \item Исследователи--- составлять карты и раскрывать тайны болот.
    \item Вестники Тьмы--- распространять Тьму, сеять разрушение и хаос.
    \item Хранители топей--- защищать болота, их тайны и местных обитателей.
  \end{itemize}
\end{description}
\vfill
\pagebreak

% Powers that be
\section*{Силы}
{\bfseries Выберите 3 Силы, которые заправляют делами в
  окрестностях. Опишите, что вас связывает с ними.} Вы имеете хорошую
Славу (6) у одной из них (ваши партнеры или покровители), и дурную
Славу (1) у еще одной (ваши враги, которые сделают все, чтобы
уничтожить вас). Со всеми остальными у вас обычная Слава (4).

\vfill
\begin{tabular}{l l l}
  Фракция & Слава & Долги \\
  \hline
  Магнаты и шляхта & & \\
  Войска и наемники & & \\
  Селяне и Болотные Братья & & \\
  Грабители и контрабандисты & & \\
  Церковь и Орден & & \\
  Тайные культы и слуги Тьмы & & \\
  Болотные духи и нечисть & & \\ 
\end{tabular}
\vfill
\pagebreak

% Lair
\section*{Логово}
\begin{description}
\item[Уровень логова:]
\item[Логово отряда:]
  \begin{itemize}
    \item Корчма или лавка в центре города
      Проблема: нежелательные соседи.
    \item Хутор, фольварк или усадьба на болотах
      Проблема: труднодоступное.
    \item Склад в у реки или мастерская в пригороде
      Проблема: две противоборствующие Силы заинтересованы в
      нем.
    \item Катакомбы под городом или руины в глуши
      Проблема: неисследованное. 
    \end{itemize}
\item[Название и описание логова:] 
\end{description}

Мастер может сделать свой
ход, связанный с проблемами логова, при любом подходящем случае.
\vfill
\pagebreak

% Lair upgrades

\section*{Улучшения логова}

Улучшения логова (выберите 2 в начале игры, и каждые 2 новых Улучшения
логова увеличивают его уровень на 1):
\begin{description}
  \item[Библиотека]--- здесь собраны книги со всего света. Если
    поискать в них ответ на интересующий вопрос, то можно сделать ход
    \textsc{Знания} с бонусом.
  \item[Зал славы]--- большой стол и коллекция трофеев, здесь можно
    сделать ход \textsc{Восприятие} (человек) и социальные ходы с бонусом.
  \item[Лаборатория]--- реторты, колбы и инструменты. Все это дает
    бонус на изготовление здесь любых алхимических смесей, зелий и
    отваров.
  \item[Схрон]--- этот небольшой схрон почти невозможно найти любыми
    средствами, здесь можно спрятать вещи (в т.ч. дукаты) общим весом
    не более 10.
  \item[Ловушки и сигнализация]--- в логово очень сложно проникнуть
    незаметно.
  \item[Место силы]--- мистическое место: гробница с призраками,
    темный обелиск, бездонный пруд\ldots Здесь можно сделать
    мистические ходы с бонусом.
  \item[Тайный вход]--- в логово можно войти и выйти полностью
    незаметно.
  \item[Лазарет]--- один персонаж в неделю может остаться здесь на
    Лечение. Он получит бонус на этот ход.
  \item[Комната для развлечений]--- один персонаж в неделю может
    остаться здесь, чтобы \textsc{Предаться страстям}. Он получит бонус на этот
    ход.
  \item[Голубятня]--- почтовые голуби держат вас в курсе событий: один
    раз в неделю выбери дружественную Силу и задай ей вопрос о
    событиях на болотах.
\end{description}
\vfill
\pagebreak

% hall of memories
\section*{Зал Славы}
Здесь можно записать имена погибших членов отряда, для потомков и
пущей науки.

\begin{tabular}{p{0.5cm} l l l l}
  &Имя&Профессия&Как жил&Как умер \\
  1.&&&& \\
  2.&&&& \\
  3.&&&& \\
  4.&&&& \\
  5.&&&& \\
  
\end{tabular}
\end{document}
